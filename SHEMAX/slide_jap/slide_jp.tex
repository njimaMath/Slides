% !TeX program = pdflatex
\documentclass{beamer}
\usetheme{metropolis}

% Japanese typesetting (pdfLaTeX)
\usepackage[T1]{CJKutf8}

\usepackage{tabularx}
\usepackage{amsmath, amssymb}    % minimum pour les maths
\usepackage{amsthm}	% permet de faire des théorèmes etc...
\usepackage{stmaryrd}
\usepackage{graphicx}
\usepackage{xcolor} % pour mettre un texte en couleur avec \textcolor ou \color

\newcommand{\maybeincludegraphics}[2][]{%
  \IfFileExists{#2}{\includegraphics[#1]{#2}}{\fbox{\texttt{#2}}}%
}



\newcommand{\IP}{{\mathbb P}}
\newcommand{\DP}{{\mathrm P}}
\newcommand{\DE}{{\mathrm E}}
\newcommand{\EEP}{\mathbb E}
\newcommand{\IE}{\mathbb E}
\newcommand{\EP}{{\mathrm E}}
\newcommand{\kU}{\mathscr{U}}

\newcommand*\Z{Z_{n}^\omega(\beta)}
\newcommand{\sZ}{\mathbf{\mathcal{Z}}}
\renewcommand{\b}{\beta}
\newcommand*\Hn{H_n^\omega}
\newcommand*\Zc{\mathcal{Z}}
\newcommand*\R{\mathbb{R}}
\newcommand*\Skn{\mathcal{S}_k^n}
\newcommand{\overbar}[1]{\mkern 1.5mu\overline{\mkern-1.5mu#1\mkern-1.5mu}\mkern 1.5mu}
\newcommand{\cvlaw}{\stackrel{{ (d)}}{\longrightarrow}}
\newcommand{\cvLone}{\stackrel{{L^1}}{\longrightarrow}}
\newcommand{\cvIP}{\stackrel{{\IP}}{\longrightarrow}}
\newcommand{\eqlaw}{\stackrel{\rm{ law}}{=}}
\newcommand*\cvLdeux{\overset{L^2}{\longrightarrow}}

\newcommand{\cvto}[2]{\underset{#1 \to #2}{\longrightarrow}}
\newcommand{\om}{\omega}
\newcommand{\rmd}{\mathrm{d}}
\def\P{{\mathbb P}}

\newcommand{\e}{\varepsilon}
\newcommand{\dd}{\mathrm{d}}
\def\P{{\mathbb P}}

\bibliographystyle{amsplain}  

\title{二次元有向ポリマーの最大値に関する問題}
\author{中島修太(慶應義塾大学)}
\institute{Cl\'ement Cosco, Ofer Zeitouni との共同研究}
\date[KPT 2004] % (optional)

\metroset{block=fill,numbering = fraction}

\begin{document}
\begin{CJK*}{UTF8}{min}
\frame{\titlepage}
\begin{frame}
\frametitle{自己紹介}
\begin{itemize}
    \item 名古屋大学 学部生 (2010-2014) <- 4年生のときに吉田先生の授業を受けた\\ 
    \item 京都大学 修士、博士  (2014-2019) <- 指導教員は福島先生 \\
    \item 2015年にComets-Fukushima-Yoshida-Nakajimaでジャンプつき有向ポリマーの論文を発表 \\
    \item 2019年から2020年まで名古屋大学でポスドク (中島-中島完成)
\end{itemize}
\end{frame}

\section{モデルと主結果}
\begin{frame}{有向ポリマーとは?}
\begin{itemize}
    \item \textbf{有向ポリマー(Directed polymer)}は,さまざまな物理現象をモデル化する:
    \begin{itemize}
        \item \textbf{無秩序環境}における\textbf{ランダムなポリマー成長}
        \item 乱流系における\textbf{界面成長}
        \item 確率的(ランダム)システムにおける\textbf{経路最適化}
    \end{itemize}
    \item \textbf{主な特徴:}
    \begin{itemize}
        \item ポリマーの経路は\textbf{特定の方向(時間方向)}に沿って進む.
        \item 環境は各配置にランダムな\textbf{エネルギー}を割り当てる.
        \item 系は\textbf{エネルギー最小化}と\textbf{エントロピー最大化}のバランスを取る.
    \end{itemize}
    \item これらは,以下のようなより広い確率構造とも結び付く:
    \begin{itemize}
        \item \textbf{分枝過程}
        \item さまざまな成長過程を記述する\textbf{KPZ普遍性クラス}
    \end{itemize}
\end{itemize}
\end{frame}

\begin{frame}
\frametitle{有向ポリマー}
\begin{itemize}
    \item $\DP_x$ を $\mathbb{Z}^d$ 上で $x$ から始まる単純ランダムウォーク(SRW)の測度とする.
    \item $(\omega(n, x))_{n\in\mathbb N,x\in \mathbb Z^d}$ を平均0の i.i.d.\ 確率変数とし,次を仮定する:
  $$\mathbb{E}[\omega(n, x)^2] = 1,\quad 
  \Lambda(\beta) := \log \mathbb{E}[e^{\beta \omega(n, x)}] < \infty,\,\forall \beta > 0.$$
\end{itemize}
\pause
\begin{figure}
    \centering
\maybeincludegraphics[width=0.75\linewidth]{rb.png}
\end{figure}
~\\
\vspace{-10mm}
{\color{red}$\bullet$}: $\omega=0$,{\color{blue}$\bullet$}: $\omega=1$
\end{frame}

\begin{frame}
\frametitle{有向ポリマー}
\begin{itemize}
    \item $\DP_x$ を $\mathbb{Z}^d$ 上で $x$ から始まる単純ランダムウォーク(SRW)の測度とする.
    \item $(\omega(n, x))_{n\in\mathbb N,x\in \mathbb Z^d}$ を平均0の i.i.d.\ 確率変数とし,次を仮定する:
  $$\mathbb{E}[\omega(n, x)^2] = 1,\quad 
  \Lambda(\beta) := \log \mathbb{E}[e^{\beta \omega(n, x)}] < \infty,\,\forall \beta > 0.$$
\end{itemize}
\begin{figure}
    \centering
\maybeincludegraphics[width=0.75\linewidth]{rbsrw.png}
\end{figure}
\end{frame}
\begin{frame}
\frametitle{有向ポリマー}
\begin{itemize}
\item $\DP_x$ は $\mathbb Z^d$ 上で $x$ から始まる単純ランダムウォークの測度,
\item $\omega(n,x),\, n\in \mathbb N,\,x\in \mathbb Z^d$ は平均0の i.i.d.\ 確率変数で
  $\IE[\omega(n,x)^2]=1$,かつ 
  $$\Lambda(\beta):=\log \IE[e^{\beta \omega(n,x)}]<\infty,\quad \forall \beta>0.$$
\end{itemize}
\begin{block}{ポリマー分配関数}
\begin{equation*} 
W_N(\beta,x):=\DE_x\left[e^{\sum_{n=1}^N \left(\beta \omega(n,S_n) -\Lambda(\beta)\right)}\right], \ x\in \mathbb Z^d.
\end{equation*} 
\end{block}
\pause
\begin{block}{ポリマー測度}
\[P^{polymer}_{N,\beta}((S_n)_{n=0}^N) := \frac{e^{\sum_{n=1}^N (\beta \omega(n,S_n) - \Lambda(\beta))}}{W_N( \beta,0)} \DP_0((S_n)_{n= 0}^N).
\]
\end{block}
\end{frame}



\begin{frame}
\frametitle{弱・強無秩序(Weak/Strong disorder)}
\begin{block}{ポリマー分配関数}
\begin{equation*} 
W_N(x) : = W_N(\beta,x):=\DE_x\left[e^{\sum_{n=1}^N \left(\beta \omega(n,S_n) -\Lambda(\beta)\right)}\right], \ x\in \mathbb Z^d.
\end{equation*} 
\end{block}
\begin{block}{相転移}
ある $\beta_c\in [0,\infty]$ が存在して:

$\b > \beta_c$ なら $W_N(\b,x) \to 0$(局在相),

$\b < \beta_c$ なら $W_N(\b,x) \to W_\infty(\b,x)>0$(拡散相).
\end{block}
\pause[2]
\begin{itemize}
\item $d=1,2$ では $\beta_c = 0$(常に局在).(Carmona-Hu 02, Comets-Shiga-Yoshida 03)
\item $d\geq 3$ では $\beta_c >0$(相転移が存在).(Comets-Yoshida 06)
\end{itemize}
\end{frame}

\begin{frame}
  \frametitle{KPZ方程式との関係}
  \textcolor{blue}{$W_N(\beta,x)$ は次の離散版を満たす:}
  \begin{block}
  {確率熱方程式(stochastic heat equation):}
  \begin{equation*} 
  \frac{\partial}{\partial t} u(t,x) = \frac12 \Delta u(t,x) +
   \beta u(t,x) \xi(t,x),
  \end{equation*}
  ここで $\xi(t,x)$ は $(t,x)\in \mathbb R_+ \times \mathbb R^d$ 上のホワイトノイズである.
  \end{block}
  \pause[2] 
  Hopf--Cole変換:$h(t,x)=\log u(t,x)$. \textcolor{blue}{($\log W_N(\beta,x)$ に対応)}
  \begin{block}
  {Kardar--Parisi--Zhang(KPZ)方程式:}
  \begin{equation*} 
  \frac{\partial}{\partial t} h(t,x) = \frac12 \Delta h(t,x) +  \frac12   |\nabla h(t,x) |^2+
   \beta  \xi(t,x).
  \end{equation*}
  \end{block}
  \pause[3]
$d\geq 2$ では,これらの方程式は(そのままでは)良定義ではない!
\end{frame}

\begin{frame}
\frametitle{中間領域}
中間領域($d=1$):
\begin{block}{Theorem (Alberts-Khanin-Quastel 14)}
Let $d=1$. For $\beta_N = \frac{\hat \beta}{N^{1/4}}$: \ $\log W_{tN}(\beta_N,x\sqrt N) \cvlaw \mathrm{KPZ}_{\hat{\beta}}(t,x)$
\end{block}
\pause[2]
中心極限定理($d=2$).
$\lambda(\hat \beta)^2 := \log \frac{1}{1-\hat{\beta}^2}$ とおく.
\begin{block}{Theorem (Caravenna-Sun-Zygouras 17, Caravenna-Cottini 22, Cosco-Donadini 23+)}
Let $d=2$. For $\beta_N = \frac{\sqrt{\pi}\hat \beta}{\sqrt{\log N}}:$
  \begin{align*}
 (i) &\ \forall \hat{\b}<1:\quad \log W_N(\beta_N,0) \cvlaw \mathcal N\left(-\frac {\lambda^2} 2,\lambda^2\right),\\
 (ii) &\ \forall \hat{\beta} \geq 1: \quad W_N(\beta_N,0)\to 0.
  \end{align*}
\end{block}
\end{frame}
\begin{frame}{中間領域}
\begin{block}{Theorem (Caravenna-Sun-Zygouras 17, Caravenna-Cottini 22, Cosco-Donadini 23+)}
Let $d=2$. For $\beta_N = \frac{\sqrt{\pi}\hat \beta}{\sqrt{\log N}}:$
  \begin{align*}
 (i) &\ \forall \hat{\b}<1:\quad \log W_N(\beta_N,0) \cvlaw \mathcal N\left(-\frac {\lambda^2} 2,\lambda^2\right),\\
 (ii) &\ \forall \hat{\beta} \geq 1: \quad W_N(\beta_N,0)\to 0.
  \end{align*}
\end{block}
\begin{itemize}
    \item 臨界点 $\hat{\beta}=1$ では,場 $(W_{tN}(\beta_N,\lfloor \sqrt N x\rfloor))_{t\geq 0,\,x\in [-1,1]^2}$ が分布の意味で,いわゆる\textbf{Critical Stochastic Heat Flow}に収束することが知られている.
\end{itemize}
\end{frame}
\begin{frame}
  \frametitle{CLTの証明(Cosco--Donadini)}
$\hat \beta<1$ とする.示したいのは $\log W_N(\beta_N,0) \cvlaw \mathcal N\left(-\frac {\lambda^2} 2,\lambda^2\right)$ である.
\begin{itemize}
  \item<1-> ステップ1(decoupling):任意の $M>0$ に対し,$W_N\underset{N\to\infty}{\approx} \prod_{k=1}^M W_{k,M}$ が $L^2$ で成り立つ.ここで
  \[W_{k,M} := \DE\left[e^{\sum_{n=N^{k/M}}^{N^{(k+1)/M}} \left(\beta_N \omega(n,S_n) - \Lambda(\beta_N)\right)}\right].\]
  \item ステップ2(Taylor展開:$\log (1+x)\approx x-x^2/2$ + CLT):
  \begin{align*}
    \sum_{k=1}^M \log W_{k,M} & \underset{M\to\infty}{\approx} \sum_{k=1}^M (W_{k,M}-1) - \frac{1}{2}(W_{k,M}-1)^2 \\
    &\underset{N\to\infty,M\to\infty}{\longrightarrow} \mathcal N(0,\lambda^2) -\frac{\lambda^2}{2}.
  \end{align*}
\end{itemize}
\end{frame}

\begin{frame}
\frametitle{対数相関場}
$\beta_N = \hat{\beta}\sqrt{\pi/\log N}$ を思い出す.次を定義する:
$$ h_N(x):=\sqrt{\log N} \left(\log W_N(\beta_N,x\sqrt N) - \IE \log W_N(\beta_N,x\sqrt{N})\right).$$
\begin{block}{Theorem (Caravenna-Sun-Zygouras 20)}
  As $N\to\infty$,
\begin{align*} \label{eq:GFFlimit}
\forall \hat{\b}<1: \ h_N(x)& \cvlaw \sqrt{\frac {\hat {\b}^2} {1-{\hat \b}^2}} G(x),
\end{align*}
with $G(x)$ a log-correlated Gaussian field on $\mathbb R^2$ independent of $\hat{\beta}$, i.e., when $|x-y|\to 0$,\ $$\IE[G(x)G(y)]\sim \log \frac{1}{|x-y|}.$$ 
\end{block} 
\end{frame}


\begin{frame}{主結果}
$h_N(x):=\sqrt{\log N} \left(\log W_N(\beta_N,x\sqrt N) - \IE \log W_N(\beta_N,x\sqrt{N})\right)$ と定義する.
\begin{block}{Cosco-N.-Zeitouni, 2025+}
  For all $\hat \beta < 1$, as $N\to\infty$,
\[\ \lim_{N\to\infty}(\log N)^{-1}\max_{|x|\leq N^{1/2}} h_N(x) = \int_{0}^1 \sigma(t) {\rm d}t,
\]
where 
$$\sigma(t) :=  \sqrt{\frac{\hat\beta^2}{1-\hat\beta^2 (1-t)}}.$$
\end{block}
\begin{itemize}
\item<2-> 方針:不均一BBMとの比較.
\item<3-> 必要:$q = O(\sqrt{\log N})$ の範囲で $\IE[(W_N(x))^q]$(delta-bose gas)の評価.
\end{itemize}
\end{frame}


\section{関連研究}

\begin{frame}{分枝ブラウン運動(BBM)}

\textbf{定義:}
\begin{itemize}
    \item \(t = 0\) に原点の1粒子から開始する.
    \item 粒子は(独立な)指数分布に従うランダム時刻まで\textbf{ブラウン運動}を行う({\color{red}指数時計}).
    \item 分枝時刻に粒子は2つの独立な粒子に分裂する.
    \item 各子孫は独立にブラウン運動する.
    \item 以降も独立な指数時計により同様に分枝を繰り返す.
    \item \( \mathcal{N}(t) \):時刻 \( t \) に生存している粒子の集合.
\end{itemize}
%\pause[2]
%\textbf{Key Properties:}
%\begin{itemize}
    %\item 
%    \item \( |\mathcal N(t)|e^{-t} \): A \textbf{martingale}, converging almost surely %to $1$.
%\end{itemize}

\end{frame}
\begin{frame}{BBMのシミュレーション}
~\\
\vspace{-10mm}
\begin{figure}
    \centering
    \maybeincludegraphics[width=0.8\linewidth]{BBM.png}
    \label{fig:enter-label}
\end{figure}    
\url{https://njima091.github.io/Lectures/miscellany/Bbm.html}
\end{frame}
\begin{frame}{分枝過程の最大値}
分枝ブラウン運動の最大値:
\begin{block}{Bramson 83, Bramson-Zeitouni 09
  }
  \[\max_{v \in \mathcal N(t)} B^v_t \approx \sqrt 2 t -\frac{3}{2\sqrt 2} \log t + X,\]
  where $X$ follows a shifted Gumbel distribution.
\end{block}
\pause[2]
同様の現象は,多くの(ガウス)対数相関場(Gaussian free field,cover times,random matrices,ゼータ関数,…)でも現れる.
\end{frame}

\begin{frame}{不均一BBM}

\textbf{定義:}
\begin{itemize}
    \item 古典的BBMの\textbf{時間依存分散}版:
    \[
    \sigma_T^2(t) := \sigma^2(t/T),
    \]
    ここで:
    \begin{itemize}
        \item \( \sigma \) は \( [0, 1] \) 上の\textbf{滑らかで狭義単調減少}な関数.
        \item \( \sigma \) の導関数は\textbf{負の定数}で上から抑えられる.
    \end{itemize}
    \item 粒子:
    \begin{itemize}
        \item 分散 \( \int_0^{\cdot} \sigma_T^2(s)\, \mathrm{d}s\) をもつ\textbf{ブラウン運動}を行う.
        \item 独立な指数時計で分枝する.
    \end{itemize}
\end{itemize}

\end{frame}
\begin{frame}{不均一BBMの最大値}
不均一分枝ブラウン運動の最大値:
\begin{block}{Feng-Zeitouni 13}
Suppose that $\sigma:[0,1]\to (0,\infty)$ is a smooth function such that $\sup_{x\in [0,1]}\sigma'(x)<0$. As $T\to\infty$, 
  \[\max_{v \in \mathcal N(T)} B^v_T = v_\sigma T - \Omega(T^{1/3}),\]
  where we define
  $$v_\sigma := \sqrt 2 \int_0^1 \sigma(s) {\rm d} s.$$
\end{block}
\pause[2]
主要項の係数や次の次数が古典的BBMとは異なる.
\end{frame}






\section{モーメント評価}
\begin{frame}
\frametitle{モーメント公式}
再掲:$W_N(\beta_N)=\DE_0\big[\exp\big\{\sum_{n=1}^N (\beta_N \omega(n,S_n) - \Lambda(\beta_N))\big\}\big]$.
\begin{block}{モーメント公式}
  $\omega(n,x)\sim \mathcal N(0,1)$ とする.このとき
\[
\IE\left[W_N^q\right] = \DE^{\otimes q}_0\left[e^{\beta_N^2\sum_{n=1}^N \sum_{1\leq i<j \leq q} \mathbf{1}_{S_n^{i} = S_n^j}} \right],
\]
ここで $(S^i_n)_{i\leq q}$ は SRW の独立なコピーである.
\end{block}
\begin{itemize}
  \item Delta-Bose gas($\delta$-Bose gas).
  \item 非ガウスの場合は公式があまりきれいにならない.
\end{itemize}
% \pause[2]
% Proof strategy: the main contribution will come from trajectories with successive pairwise interactions.
\end{frame}

\begin{frame}
\frametitle{上からの評価}
$W_N = W_N(\beta_N,0)$ と書き,$\beta_N =\hat{\beta}\sqrt{\pi/\log N}$ とおく.
\begin{block}{Theorem (Cosco-Zeitouni 22)}
Assume $\omega(i,x) \sim \mathcal N(0,1)$. For any $\hat \beta$ small enough, if $\binom q 2\leq \textcolor{blue}{\frac{1}{3}}\frac{\hat \beta^2}{1-\hat \beta^2}\log N$, then we have 
\[\IE\left[(W_N)^q\right] \leq e^{\lambda^2 \binom{q}{2} + o(q^2)},
\]
where $\lambda^2:=\lambda(\hat \beta)^2 := \log \frac{1}{1-\hat{\beta}^2}$.
\end{block}
\begin{itemize}
\item<2-> $q^2=o(\log N / \log \log N)$ なら,全ての $\hat \beta <1$ に対して成り立つ.
\item<3-> $q\in \mathbb N$ を固定すると,$\hat \beta < 1$ の領域で $W_N$ は全てのモーメントをもつ(Lygkonis--Zygouras 21).
\item<4-> 改良点:$\hat{\beta_0} \to 1$,$1/3\to 1$,および $o(q^2) \to o(1)$.
\end{itemize}
\end{frame}






\begin{frame}
  \frametitle{改良}
$t\leq 1$ に対して次を定義する:
  $$
  W_{tN}:=\DE_0\left[e^{\sum_{n=1}^{tN} \left(\beta_N \omega(n,S_n) -\Lambda(\beta_N)\right)}\right],\quad \lambda_{t}^2 := \log \left(\frac{1}{1-\hat{\beta}^2 t }\right).$$

  \begin{block}{Theorem (Cosco-N, 2025+)}
For any $\hat \beta < 1$, $\alpha< 1$, and $t\leq 1$, if  $ \binom q 2\leq \alpha \frac{\hat \beta^2}{1-t \hat \beta^2} \log N$, 
  \[\IE\left[(W_{tN})^q\right] \leq e^{\lambda_{t}^2 \binom{q}{2} + o(q^2)},
  \]
  where $o(q^2)$ depends only on $N,\alpha$.
  \end{block}
  \fbox{\textbf{ポイント}}
  \begin{itemize}
      \item 有限回の交差を許す
      %\item Analogous formula to Chen's chaos expansion
      \item 一般の重み
  \end{itemize}
\end{frame}




\begin{frame}
  \frametitle{下からの評価}
  \begin{block}{Theorem (Cosco-Zeitouni 23)}
    (i) For any $\hat{\beta} <1$, $\binom q 2=O(\log N)$, 
    \[\IE\left[W_N^q\right] \geq e^{\lambda^2 \binom{q}{2} - o(q^2)}.
    \]
    (ii) If $\binom q 2 \geq (\log N)^2$, $\IE[W_N^q] \geq e^{cN/\log N}$.
    \end{block}
    % \begin{itemize}
    %   \item Different argument: decoupling on dyadic times
    % \end{itemize}
\end{frame}

\section{証明方針}
\iffalse
\begin{frame}{Proof Strategy (Overview)}
\small
Our proof strategy to approximate $ \log W_N(x)$ by an \emph{inhomogeneous BBM} approach involves the following steps:
\begin{itemize}
    \item \textbf{Hierarchical Decomposition:} Decompose $\log W_N(x)$ into (approximately) independent components corresponding to different time intervals.
    \item \textbf{Chaining:} Analyze correlations and independence between these components based on spatial and temporal scales.
    \item \textbf{Variance and Log-Correlation:} Relate the variance of these components to a log-correlated Gaussian structure.
    \item \textbf{Comparison with Inhomogeneous BBM:} Show that the hierarchical decomposition mirrors the structure of inhomogeneous BBM, allowing us to use known results on the maximum of such branching systems.
\end{itemize}
Each step is detailed in the following slides.
\end{frame}
\fi
\begin{frame}{階層分解}
\small
$\log W_N(x)$ を次のように分解する:
\[
    \log W_N(x) \;=\; \sum_{k=1}^K \log W_{N,k}(x),
\]
ここで各成分は,ある時間区間に対応する分配関数で:
\[
    W_{N,k}(x) 
    :=\; 
    \frac{W_{N^{(k+1)/K}}(x)}{W_{N^{k/K}}(x)}.
\]
この分解は,次のスライドで見るような\emph{階層構造}を示唆している.
\end{frame}

\begin{frame}{Chaining argument}
\small
空間・時間方向の相関を次のように評価する:
\begin{enumerate}
    \item \textbf{近い点:} $|x-y| \le N^{k/(2K) - \epsilon}$ なら
\[
       \log W_{N,k}(x) \;\approx\; \log W_{N,k}(y).
\]
    \item \textbf{遠い点:} $|x-y| \ge N^{k/(2K) + \epsilon}$ なら
\[
       \log W_{N,k}(x) \;\perp\; \log W_{N,k}(y).
\]
    \item \textbf{異なるレベル:} $k_1 \neq k_2$ のとき,$\log W_{N,k_1}(x)$ と $\log W_{N,k_2}(y)$ は近似的に独立.
\end{enumerate}
これらは分枝過程に現れる階層的な相関構造を表す: すなわち
$$\text{$|x-y| \approx N^{k/(2K)}$ のとき,分岐時刻は $N^{k/K}$ に対応する.}$$
\end{frame}

\begin{frame}{分散と対数相関}
\small
各成分の分散は,関数 $f(t)$ により近似的に
\[
    \mathrm{Var}\bigl(\log W_{N,k}(x)\bigr) 
    \;\sim\; 
    \frac{f\bigl(\tfrac{k}{K}\bigr)}{K \,\log N},
    \quad
    \text{where}
    \quad
    f(t) \;=\; \frac{\hat{\beta}^2}{1 - \hat{\beta}^2\,t}.
\]
$k$ で足し上げると,場
\[
    h_N(x) := \sqrt{\log N}\,(\log W_N(\sqrt{N} x)-\IE \log W_N(\sqrt{N} x))
\]
は\emph{対数相関}ガウス場になる:
\[
    \mathrm{Cov}\bigl(h_N(x), h_N(y)\bigr)
    \;\approx\; 
    \int_{-2 \log |x-y|}^1 f(t) \, \mathrm{d}t,
\]
これは $N\to\infty$ で不均一BBMの挙動を反映している.
\end{frame}

\begin{frame}{不均一BBMとの比較}
\small
不均一分枝ブラウン運動(BBM)との類比により:
\begin{itemize}
    \item 分解の「レベル」$k$ は,時間依存の分枝イベントに対応する.
    \item 分散プロファイル $f(t)$ は,BBMの時間依存拡散係数に対応する.
    \item 不均一BBMの最大値に関する既知の結果(例:Feng--Zeitouni)より
\[
        \max_{|x|\le 1} \;h_N(x)
        \;=\;
        \max_{|x|\le 1} \;\sqrt{\log N}\,\log W_N(x)
        \;\longrightarrow\;
        \int_0^1 \sqrt{f(t)}\,\mathrm{d}t.
\]
\end{itemize}
従って, $h_N(x)$ の最大値は,不均一BBMの最大値解析と並行する方法で導ける.

{\color{red}{厳密な証明では,十分良い近似のためにモーメント評価に関する鋭い条件が必要になる.}}
\end{frame}

\section{今後の課題}

\begin{frame}


\frametitle{今後の課題}
\begin{itemize}
\item<1-> $\hat \beta = 1$:臨界2次元 Stochastic Heat Flow\\
(Caravenna--Sun--Zygouras 23,Liu--Zygouras 24+)
\end{itemize}
\end{frame}
\begin{frame}{最後に}


{\Large  吉田先生,ご還暦おめでとうございます.今後のご研究,そしてご執筆のますますのご発展を心よりお祈り申し上げます.}
\end{frame}
\end{CJK*}
\end{document}
