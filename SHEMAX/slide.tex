\documentclass{beamer}
\usetheme{metropolis}
\usepackage{tabularx}
\usepackage{amsmath, amssymb}    % minimum pour les maths
\usepackage{amsthm}	%permet de faire des théorèmes etc...
\usepackage{ stmaryrd }
\usepackage[utf8]{inputenc}   
\usepackage[T1]{fontenc}
\usepackage{ruby}


\usepackage[utf8]{inputenc}		% ou latin1 ou macroman à la place de utf8, pour encodage des caractères
\usepackage[T1]{fontenc}		 %encodage des caractères plus évolué (accents dans le pdf,polices)

\usepackage{graphicx}


\usepackage{xcolor} % pour mettre un texte en couleur avec \textcolor ou \color



\newcommand{\IP}{{\mathbb P}}
\newcommand{\DP}{{\mathrm P}}
\newcommand{\DE}{{\mathrm E}}
\newcommand{\EEP}{\mathbb E}
\newcommand{\IE}{\mathbb E}
\newcommand{\EP}{{\mathrm E}}
\newcommand{\kU}{\mathscr{U}}

\newcommand*\Z{Z_{n}^\omega(\beta)}
\newcommand{\sZ}{\mathbf{\mathcal{Z}}}
\renewcommand{\b}{\beta}
\newcommand*\Hn{H_n^\omega}
\newcommand*\Zc{\mathcal{Z}}
\newcommand*\R{\mathbb{R}}
\newcommand*\Skn{\mathcal{S}_k^n}
\newcommand{\overbar}[1]{\mkern 1.5mu\overline{\mkern-1.5mu#1\mkern-1.5mu}\mkern 1.5mu}
\newcommand{\cvlaw}{\stackrel{{ (d)}}{\longrightarrow}}
\newcommand{\cvLone}{\stackrel{{L^1}}{\longrightarrow}}
\newcommand{\cvIP}{\stackrel{{\IP}}{\longrightarrow}}
\newcommand{\eqlaw}{\stackrel{\rm{ law}}{=}}
\newcommand*\cvLdeux{\overset{L^2}{\longrightarrow}}

\newcommand{\cvto}[2]{\underset{#1 \to #2}{\longrightarrow}}
\newcommand{\om}{\omega}
\newcommand{\rmd}{\mathrm{d}}
\def\P{{\mathbb P}}

\newcommand{\e}{\varepsilon}
\newcommand{\dd}{\mathrm{d}}
\def\P{{\mathbb P}}

\bibliographystyle{amsplain}  

\title{The Maximum of 2D Directed Polymers in the Subcritical Regime and Inhomogeneous Branching Brownian Motion}
\author{Shuta Nakajima (Meiji University)}
\institute{ Joint work with Cl\'ement Cosco and Ofer Zeitouni}
\date[KPT 2004] % (optional)

\metroset{block=fill,numbering = fraction}

\begin{document}
\frame{\titlepage}
\begin{frame}{Summary of the Talk}
\begin{itemize}
    \item This talk examines the behavior of the maximum free energy of 2D directed polymers in the subcritical regime.
    \item The main result identifies the leading coefficient of this maximum, which is linked to Inhomogeneous Branching Brownian Motion (Inhomogeneous BBM).
    \item We expect that this work lays the groundwork for further understanding the intermediate regime of directed polymers through BBM-based approaches.
\end{itemize}
\end{frame}

\section{Model and Main Results}
\begin{frame}{What are Directed polymers?}
\begin{itemize}
    \item \textbf{Directed polymers} model various physical phenomena, including:
    \begin{itemize}
        \item \textbf{Random polymer growth} in disordered environments.
        \item \textbf{Interface growth} in turbulent systems.
        \item \textbf{Path optimization} in stochastic or random systems.
    \end{itemize}
    \item \textbf{Key Features:}
    \begin{itemize}
        \item The polymer path evolves \textbf{in a specific direction}.
        \item The environment assigns random \textbf{energy values} to different configurations.
        \item The system seeks a balance between \textbf{energy minimization} and \textbf{entropy maximization}.
    \end{itemize}
    \item These models connect to broader probabilistic structures, e.g.,
    \begin{itemize}
        \item \textbf{Branching processes.}
        \item \textbf{KPZ universality class}, which describes various growth processes.
    \end{itemize}
\end{itemize}
\end{frame}

\begin{frame}
\frametitle{Directed polymers}
\begin{itemize}
    \item Let $\DP_x$ denote the measure of SRW starting at $x$ on $\mathbb{Z}^d$.
    \item Let $(\omega(n, x))_{n\in\mathbb N,x\in \mathbb Z^d}$  be centered i.i.d.\ RVs with
  $$\mathbb{E}[\omega(n, x)^2] = 1,\quad 
  \Lambda(\beta) := \log \mathbb{E}[e^{\beta \omega(n, x)}] < \infty,\,\forall \beta > 0.$$
\end{itemize}
\pause
\begin{figure}
    \centering
\includegraphics[width=0.75\linewidth]{rb.png}
\end{figure}
~\\
\vspace{-10mm}
{\color{red}$\bullet$}: $\omega=0$; {\color{blue}$\bullet$}: $\omega=1$
\end{frame}

\begin{frame}
\frametitle{Directed polymers}
\begin{itemize}
    \item Let $\DP_x$ denote the measure of SRW starting at $x$ on $\mathbb{Z}^d$.
    \item Let $(\omega(n, x))_{n\in\mathbb N,x\in \mathbb Z^d}$  be centered i.i.d.\ RVs with
  $$\mathbb{E}[\omega(n, x)^2] = 1,\quad 
  \Lambda(\beta) := \log \mathbb{E}[e^{\beta \omega(n, x)}] < \infty,\,\forall \beta > 0.$$
\end{itemize}
\begin{figure}
    \centering
\includegraphics[width=0.75\linewidth]{rbsrw.png}
\end{figure}
\end{frame}
\begin{frame}
\frametitle{Directed polymers}
\begin{itemize}
\item $\DP_x$ is the measure of simple random walk started at $x$ on $\mathbb Z^d$,
\item $\omega(n,x), n\in \mathbb N,x\in \mathbb Z^d$ are centered i.i.d.\ random variables with $\IE[\omega(n,x)^2]=1$,  $\Lambda(\beta):=
\log \IE[e^{\beta \omega(n,x)}]<\infty$ for all $\beta>0$.
\end{itemize}
\begin{block}{Polymer partition function:}
\begin{equation*} 
W_N(\beta,x):=\DE_x\left[e^{\sum_{n=1}^N \left(\beta \omega(n,S_n) -\Lambda(\beta)\right)}\right], \ x\in \mathbb Z^d.
\end{equation*} 
\end{block}
\pause
\begin{block}{Polymer measure:}
\[P^{polymer}_{N,\beta}((S_n)_{n=0}^N) := \frac{e^{\sum_{n=1}^N (\beta \omega(n,S_n) - \Lambda(\beta))}}{W_N( \beta,0)} \DP_0((S_n)_{n= 0}^N).
\]
\end{block}
\pause
Reference: Comets - Saint-Flour notes
\end{frame}



\begin{frame}
\frametitle{Weak/Strong disorder}
\begin{block}{Polymer partition function:}
\begin{equation*} 
W_N(x) : = W_N(\beta,x):=\DE_x\left[e^{\sum_{n=1}^N \left(\beta \omega(n,S_n) -\Lambda(\beta)\right)}\right], \ x\in \mathbb Z^d.
\end{equation*} 
\end{block}
\begin{block}{Phase transition:}
There exists $\beta_c\in [0,\infty]$ such that:

$W_N(\beta,x) \to 0$ if $\b > \beta_c$ (localized phase),

$W_N(\b,x) \to W_\infty(\b,x)>0$ if $\b < \beta_c$ (diffusive phase).
\end{block}
\pause[2]
\begin{itemize}
\item If $d=1,2$, $\b_c = 0$ (always localized),
\item If $d\geq 3$, $\b_c >0$ (phase transition).
\end{itemize}
\end{frame}

\begin{frame}
  \frametitle{Link to the KPZ equation}
  \textcolor{blue}{$W_N(\beta,x)$ satisfies a discrete version of}
  \begin{block}
  {The stochastic heat equation:}
  \begin{equation*} 
  \frac{\partial}{\partial t} u(t,x) = \frac12 \Delta u(t,x) +
   \beta u(t,x) \xi(t,x),
  \end{equation*}
  with $\xi(t,x)$ a white noise on $(t,x)\in \mathbb R_+ \times \mathbb R^d$.
  \end{block}
  \pause[2] 
  The Hopf-Cole transform: $h(t,x)=\log u(t,x)$. \textcolor{blue}{($\log W_N(\beta,x)$)}
  \begin{block}
  {The Kardar-Parisi-Zhang (KPZ) equation:}
  \begin{equation*} 
  \frac{\partial}{\partial t} h(t,x) = \frac12 \Delta h(t,x) +  \frac12   |\nabla h(t,x) |^2+
   \beta  \xi(t,x).
  \end{equation*}
  \end{block}
  \pause[3]
The two equations are not well defined when $d\geq 2$!
\end{frame}

\begin{frame}
\frametitle{Intermediate regime}
Intermediate regime ($d=1$):
\begin{block}{Theorem (Alberts-Khanin-Quastel 14)}
Let $d=1$. For $\beta_N = \frac{\hat \beta}{N^{1/4}}$: \ $\log W_{tN}(\beta_N,x\sqrt N) \cvlaw \mathrm{KPZ}_{\hat{\beta}}(t,x)$
\end{block}
\pause[2]
Central Limit Theorem ($d=2$).
Let $\lambda(\hat \beta)^2 := \log \frac{1}{1-\hat{\beta}^2}$.
\begin{block}{Theorem (Caravenna-Sun-Zygouras 17, Caravenna-Cottini 22, Cosco-Donadini 23+)}
Let $d=2$. For $\beta_N = \frac{\sqrt{\pi}\hat \beta}{\sqrt{\log N}}:$
  \begin{align*}
 (i) &\ \forall \hat{\b}<1:\quad \log W_N(\beta_N,0) \cvlaw \mathcal N\left(-\frac {\lambda^2} 2,\lambda^2\right),\\
 (ii) &\ \forall \hat{\beta} \geq 1: \quad W_N(\beta_N,0)\to 0.
  \end{align*}
\end{block}
\end{frame}
\begin{frame}{Intermediate regime}
\begin{block}{Theorem (Caravenna-Sun-Zygouras 17, Caravenna-Cottini 22, Cosco-Donadini 23+)}
Let $d=2$. For $\beta_N = \frac{\sqrt{\pi}\hat \beta}{\sqrt{\log N}}:$
  \begin{align*}
 (i) &\ \forall \hat{\b}<1:\quad \log W_N(\beta_N,0) \cvlaw \mathcal N\left(-\frac {\lambda^2} 2,\lambda^2\right),\\
 (ii) &\ \forall \hat{\beta} \geq 1: \quad W_N(\beta_N,0)\to 0.
  \end{align*}
\end{block}
\begin{itemize}
    \item At criticality, $\hat{\beta}=1$, it is known that the field $(W_{tN}(\beta_N,\lfloor \sqrt N x\rfloor))_{t\geq 0,\,x\in [-1,1]^2}$ converges to the so-called \textbf{Critical Stochastic Heat Flow} in a distributional sense.
\end{itemize}
\end{frame}
\begin{frame}
  \frametitle{Proof of the CLT (Cosco-Donadini)}
Let $\hat \beta<1$. We want to prove $\log W_N(\beta_N,0) \cvlaw \mathcal N\left(-\frac {\lambda^2} 2,\lambda^2\right).$
\begin{itemize}
  \item<1-> Step 1 (decoupling): $\forall M>0$, $W_N\underset{N\to\infty}{\approx} \prod_{k=1}^M W_{k,M}$ in $L^2$, where 
  \[W_{k,M} := \DE\left[e^{\sum_{n=N^{k/M}}^{N^{(k+1)/M}} \left(\beta_N \omega(n,S_n) - \Lambda(\beta_N)\right)}\right].\]
  \item Step 2 ([Taylor: $\log (1+x)\approx x-x^2/2$] + CLT): 
  \begin{align*}
    \sum_{k=1}^M \log W_{k,M} & \underset{M\to\infty}{\approx} \sum_{k=1}^M (W_{k,M}-1) - \frac{1}{2}(W_{k,M}-1)^2 \\
    &\underset{N\to\infty,M\to\infty}{\longrightarrow} \mathcal N(0,\lambda^2) -\frac{\lambda^2}{2}.
  \end{align*}
\end{itemize}
\end{frame}

\begin{frame}
\frametitle{Log-correlated field}
Recall $\beta_N = \hat{\beta}\sqrt{\pi/\log N}$. We define
$$ h_N(x):=\sqrt{\log N} \left(\log W_N(\beta_N,x\sqrt N) - \IE \log W_N(\beta_N,x\sqrt{N})\right).$$
\begin{block}{Theorem (Caravenna-Sun-Zygouras 20)}
  As $N\to\infty$,
\begin{align*} \label{eq:GFFlimit}
\forall \hat{\b}<1: \ h_N(x)& \cvlaw \sqrt{\frac {\hat {\b}^2} {1-{\hat \b}^2}} G(x),
\end{align*}
with $G(x)$ a log-correlated Gaussian field on $\mathbb R^2$ independent of $\hat{\beta}$, i.e., when $|x-y|\to 0$,\ $$\IE[G(x)G(y)]\sim \log \frac{1}{|x-y|}.$$ 
\end{block} 
\end{frame}


\begin{frame}{Main result}
Define $h_N(x):=\sqrt{\log N} \left(\log W_N(\beta_N,x\sqrt N) - \IE \log W_N(\beta_N,x\sqrt{N})\right)$.
\begin{block}{Cosco-N.-Zeitouni, 2025+}
  For all $\hat \beta < 1$, as $N\to\infty$,
\[\ \lim_{N\to\infty}(\log N)^{-1}\max_{|x|\leq N^{1/2}} h_N(x) = \int_{0}^1 \sigma(t) {\rm d}t,
\]
where 
$$\sigma(t) :=  \sqrt{\frac{\hat\beta^2}{1-\hat\beta^2 (1-t)}}.$$
\end{block}
\begin{itemize}
\item<2-> Strategy: $\text{Comparison to Inhomogeneous BBM}$.
\item<3-> Need estimates for $\IE[(W_N(x))^q]$ (delta-bose gas) with $q = O(\sqrt{\log N})$.
\end{itemize}
\end{frame}


\section{Related research}

\begin{frame}{Branching Brownian Motion (BBM)}

\textbf{Definition:}
\begin{itemize}
    \item Starts with a single particle at the origin at \( t = 0 \).
    \item The particle performs \textbf{Brownian motion} until a (independent) random time, distributed exponentially {\color{red}(exponential clock)}.
    \item At the random branching time, the particle splits into two independent particles.
        \item Each offspring 
      performs independent Brownian motion.
            \item It continues branching in the same manner with independent exponential clocks.
            \item \( \mathcal{N}(t) \): Set of particles alive at time \( t \).
\end{itemize}
%\pause[2]
%\textbf{Key Properties:}
%\begin{itemize}
    %\item 
%    \item \( |\mathcal N(t)|e^{-t} \): A \textbf{martingale}, converging almost surely %to $1$.
%\end{itemize}

\end{frame}
\begin{frame}{Simulation of BBM}
~\\
\vspace{-10mm}
\begin{figure}
    \centering
    \includegraphics[width=0.8\linewidth]{BBM.png}
    \label{fig:enter-label}
\end{figure}    
https://njima091.github.io/Lectures/miscellany/Bbm.html
\end{frame}
\begin{frame}{Maximum of Branching processes}
Maximum of Branching Brownian Motion:
\begin{block}{Bramson 83, Bramson-Zeitouni 09
  }
  \[\max_{v \in \mathcal N(t)} B^v_t \approx \sqrt 2 t -\frac{3}{2\sqrt 2} \log t + X,\]
  where $X$ follows a shifted Gumbel distribution.
\end{block}
\pause[2]
Similar phenomena for a wide range of (Gaussian) log-correlated fields (Gaussian free field, cover times, random matrices, Zeta function,...)
\end{frame}

\begin{frame}{Inhomogeneous BBM}

\textbf{Definition:}
\begin{itemize}
    \item A variation of classical BBM with \textbf{time-dependent variance}:
    \[
    \sigma_T^2(t) := \sigma^2(t/T),
    \]
    where:
    \begin{itemize}
        \item \( \sigma \) is a \textbf{smooth, strictly decreasing function} on \( [0, 1] \).
        \item The derivative of \( \sigma \) is bounded above by a \textbf{strictly negative constant}.
    \end{itemize}
    \item Particles:
    \begin{itemize}
        \item Perform \textbf{Brownian motion} with variance \( \int_0^{\cdot} \sigma_T^2(s) ds\).
        \item Branch with independent exponential clocks.
    \end{itemize}
\end{itemize}

\end{frame}
\begin{frame}{Maximum of Inhomogeneous BBM}
Maximum of Inhomogeous Branching Brownian Motion:
\begin{block}{Feng-Zeitouni 13}
Suppose that $\sigma:[0,1]\to (0,\infty)$ is a smooth function such that $\sup_{x\in [0,1]}\sigma'(x)<0$. As $T\to\infty$, 
  \[\max_{v \in \mathcal N(T)} B^v_T = v_\sigma T - \Omega(T^{1/3}),\]
  where we define
  $$v_\sigma := \sqrt 2 \int_0^1 \sigma(s) {\rm d} s.$$
\end{block}
\pause[2]
Different leading coefficients and sub-leading orders.
\end{frame}






\section{Moment estimates}
\begin{frame}
\frametitle{The moment formula}
Recall: $W_N(\beta_N)=\DE_0\big[\exp\big\{\sum_{n=1}^N (\beta_N \omega(n,S_n) - \Lambda(\beta_N))\big\}\big]$.
\begin{block}{The moment formula}
  Let $\omega(n,x)\sim \mathcal N(0,1)$. We have
\[
\IE\left[W_N^q\right] = \DE^{\otimes q}_0\left[e^{\beta_N^2\sum_{n=1}^N \sum_{1\leq i<j \leq q} \mathbf{1}_{S_n^{i} = S_n^j}} \right],
\]
where $(S^i_n),i\leq q$ are independent copies of the SRW.
\end{block}
\begin{itemize}
  \item Delta-Bose gas.
  \item The formula for non-Gaussian is less nice.
\end{itemize}
% \pause[2]
% Proof strategy: the main contribution will come from trajectories with successive pairwise interactions.
\end{frame}

\begin{frame}
\frametitle{Upper bound}
We write $W_N = W_N(\beta_N,0)$ with $\beta_N =\hat{\beta}\sqrt{\pi/\log N}$.
\begin{block}{Theorem (Cosco-Zeitouni 22)}
Assume $\omega(i,x) \sim \mathcal N(0,1)$. For any $\hat \beta$ small enough, if $\binom q 2\leq \textcolor{blue}{\frac{1}{3}}\frac{\hat \beta^2}{1-\hat \beta^2}\log N$, then we have 
\[\IE\left[(W_N)^q\right] \leq e^{\lambda^2 \binom{q}{2} + o(q^2)},
\]
where $\lambda^2:=\lambda(\hat \beta)^2 := \log \frac{1}{1-\hat{\beta}^2}$.
\end{block}
\begin{itemize}
\item<2-> if $q^2=o(\log N / \log \log N)$, then it holds for all $\hat \beta <1$,
\item<3-> fixing $q\in \mathbb N$: $W_N$ admits all moments in the region $\hat \beta < 1$. (Lygkonis-Zygouras 21)
\item<4-> To improve: $\hat{\beta_0} \to 1$ and $1/3\to 1$, $o(q^2) \to o(1)$.
\end{itemize}
\end{frame}






\begin{frame}
  \frametitle{Improvement}
For $t\leq 1$,  we define
  $$
  W_{tN}:=\DE_0\left[e^{\sum_{n=1}^{tN} \left(\beta_N \omega(n,S_n) -\Lambda(\beta_N)\right)}\right],\quad \lambda_{t}^2 := \log \left(\frac{1}{1-\hat{\beta}^2 t }\right).$$

  \begin{block}{Theorem (Cosco-N, 2025+)}
For any $\hat \beta < 1$, $\alpha< 1$, and $t\leq 1$, if  $ \binom q 2\leq \alpha \frac{\hat \beta^2}{1-t \hat \beta^2} \log N$, 
  \[\IE\left[(W_{tN})^q\right] \leq e^{\lambda_{t}^2 \binom{q}{2} + o(q^2)},
  \]
  where $o(q^2)$ depends only on $N,\alpha$.
  \end{block}
  \fbox{\textbf{Key points}}
  \begin{itemize}
      \item Allow finite intersections
      %\item Analogous formula to Chen's chaos expansion
      \item General weights
  \end{itemize}
\end{frame}




\begin{frame}
  \frametitle{Lower bound}
  \begin{block}{Theorem (Cosco-Zeitouni 23)}
    (i) For any $\hat{\beta} <1$, $\binom q 2=O(\log N)$, 
    \[\IE\left[W_N^q\right] \geq e^{\lambda^2 \binom{q}{2} - o(q^2)}.
    \]
    (ii) If $\binom q 2 \geq (\log N)^2$, $\IE[W_N^q] \geq e^{cN/\log N}$.
    \end{block}
    % \begin{itemize}
    %   \item Different argument: decoupling on dyadic times
    % \end{itemize}
\end{frame}

\section{Proof strategies}
\iffalse
\begin{frame}{Proof Strategy (Overview)}
\small
Our proof strategy to approximate $ \log W_N(x)$ by an \emph{inhomogeneous BBM} approach involves the following steps:
\begin{itemize}
    \item \textbf{Hierarchical Decomposition:} Decompose $\log W_N(x)$ into (approximately) independent components corresponding to different time intervals.
    \item \textbf{Chaining:} Analyze correlations and independence between these components based on spatial and temporal scales.
    \item \textbf{Variance and Log-Correlation:} Relate the variance of these components to a log-correlated Gaussian structure.
    \item \textbf{Comparison with Inhomogeneous BBM:} Show that the hierarchical decomposition mirrors the structure of inhomogeneous BBM, allowing us to use known results on the maximum of such branching systems.
\end{itemize}
Each step is detailed in the following slides.
\end{frame}
\fi
\begin{frame}{Hierarchical Decomposition}
\small
We decompose $\log W_N(x)$ into components:
\[
    \log W_N(x) \;=\; \sum_{k=1}^K \log W_{N,k}(x),
\]
where each component corresponds to the partition function over a time interval:
\[
    W_{N,k}(x) 
    :=\; 
    \frac{W_{N^{(k+1)/K}}(x)}{W_{N^{k/K}}(x)}.
\]
This decomposition captures a \emph{hierarchical} structure as in the next slide.
\end{frame}

\begin{frame}{Chaining argument}
\small
We analyze the correlations in space and time:
\begin{enumerate}
    \item \textbf{Close Points:} If $|x-y| \le N^{k/(2K) - \epsilon}$, then
\[
       \log W_{N,k}(x) \;\approx\; \log W_{N,k}(y).
\]
    \item \textbf{Far Points:} If $|x-y| \ge N^{k/(2K) + \epsilon}$, then
\[
       \log W_{N,k}(x) \;\perp\; \log W_{N,k}(y).
\]
    \item \textbf{Different Levels:} For $k_1 \neq k_2$, the components $\log W_{N,k_1}(x)$ and $\log W_{N,k_2}(y)$ are approximately independent.
\end{enumerate}
These properties replicate the hierarchical correlation structure seen in branching processes, i.e., $$\text{for $|x-y| \approx N^{k/(2K)}$, they split at time $N^{k/K}$.}$$
\end{frame}

\begin{frame}{Variance and Log-Correlation}
\small
Each component’s variance is approximately given by the function $f(t)$:
\[
    \mathrm{Var}\bigl(\log W_{N,k}(x)\bigr) 
    \;\sim\; 
    \frac{f\bigl(\tfrac{k}{K}\bigr)}{K \,\log N},
    \quad
    \text{where}
    \quad
    f(t) \;=\; \frac{\hat{\beta}^2}{1 - \hat{\beta}^2\,t}.
\]
Summing over $k$, the field 
\[
    h_N(x) := \sqrt{\log N}\,(\log W_N(\sqrt{N} x)-\IE \log W_N(\sqrt{N} x))
\]
becomes a \emph{log-correlated} Gaussian field:
\[
    \mathrm{Cov}\bigl(h_N(x), h_N(y)\bigr)
    \;\approx\; 
    \int_{-2 \log |x-y|}^1 f(t) \, \mathrm{d}t,
\]
mirroring the behavior of inhomogeneous BBM in the large-$N$ limit.
\end{frame}

\begin{frame}{Comparison with Inhomogeneous BBM}
\small
By analogy with inhomogeneous Branching Brownian Motion (BBM):
\begin{itemize}
    \item The “levels” $k$ in our decomposition resemble time-dependent branching events.
    \item The variance profile $f(t)$ plays the role of a time-dependent diffusion coefficient in BBM.
    \item Known results (e.g.\ Feng-Zeitouni) on inhomogeneous BBM maxima show that
\[
        \max_{|x|\le 1} \;h_N(x)
        \;=\;
        \max_{|x|\le 1} \;\sqrt{\log N}\,\log W_N(x)
        \;\longrightarrow\;
        \int_0^1 \sqrt{f(t)}\,\mathrm{d}t.
\]
\end{itemize}
Hence, the maximum of our field $h_N(x)$ can be derived via methods parallel to the analysis of maxima in inhomogeneous BBM.

{\color{red}{In a rigorous proof, a sufficiently good approximation needs a sharp condition for moment estimates.}}
\end{frame}

\section{Future works}

\begin{frame}


\frametitle{What's left}
\begin{itemize}
\item<1-> Inhomogeneous Gaussian Multiplicative Chaos.
\item<2-> $\hat \beta = 1$: The critical 2d Stochastic Heat Flow (Caravenna-Sun-Zygouras 23, Liu-Zygouras 24+)
\item<3-> What is the optimal threshold $q^2<c(\hat \beta)\log N$ in moment estimates? 
\item<4-> $o(q^2)\to o(1)$ in moment estimates \\
\qquad\qquad\qquad\qquad ($\Rightarrow$ sub-leading order?).
\end{itemize}
\end{frame}
\end{document}
